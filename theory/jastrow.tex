\documentclass{article}

\title{Jastrow Factor}

\author{Jesse van Rhijn}

\date{2019-02-20}

\usepackage{amsmath}

\begin{document}

\maketitle

The Jastrow factor employed in this library has the functional form

\begin{align}
  J = \exp(f_{ee})
\end{align}

with 

\begin{align}
  f_{ee}(R_{ij}) = \sum_{i=1}^{N_e}\sum_{j > i}^{N_e} 
    \left( \frac{b_1 R_{ij}}{1 + b_2 R_{ij}} + \sum_{p=2}^{N_b - 1} b_{p+1}R_{ij}^p \right).
\end{align}

Here, $R_{ij} = \frac{1 - \exp(-\kappa r_{ij}}{\kappa}$ are scaled distance
variables. $\kappa$ can be taken so that the exponentiated terms are
numerically well-behaved. The gradient of $f_{ee}$ is then

\begin{align}
  \frac{\partial f_{ee}}{\partial\mathbf{r}_k} = \sum_{l=1}^{N_e}
    \frac{\partial R_{kl}}{\partial\mathbf{r}_{k}}
    \frac{\partial f_{ee}}{\partial R_{kl}}.
\end{align}

We have

\begin{align}
  \frac{\partial R_{kl}}{\partial\mathbf{r}_k} = -\frac{\mathbf{r}_{kl}}{r_{kl}} e^{-\kappa r_{kl}},
\end{align}

and,

\begin{align}
  \frac{\partial f_{ee}}{\partial R_{kl}} = \frac{b_1}{(1 + b_2 R_{kl})^2} + \sum_{p=1}^{N_e - 1}
    pb_{p+1}R_{kl}^{p-1}.
\end{align}

Putting it all together, the gradient of $f_{ee}$ is

\begin{align}
  \frac{\partial f_{ee}}{\partial \mathbf{r}_k} = \sum_{l=1}^{N_e} \left( \hat{r}_{kl}e^{-\kappa r_{kl}} 
    \left( \frac{b_1}{(1 + b_2R_{kl})^2} + \sum_{p=2}^{N_b - 1} pb_{p+1}R_{kl}^{p-1}\right)\right).
\end{align}

As for the laplacian, we have

\begin{align}
  \frac{\partial^2 f_{ee}}{\partial \mathbf{r}_k^2} = \frac{\partial}{\partial \mathbf{r}_k} 
    \bullet \frac{\partial f_{ee}}{\partial \mathbf{r}_k}.
\end{align}

This divergence can be decomposed as follows. Setting

\begin{align}
  g(r_{kl}) = e^{-\kappa r_{kl}} \left( \frac{b_1}{(1 + b_2 R_{kl})^2} 
    + \sum_{p=2}^{N_b - 1}pb_{p+1}R_{kl}^{p-1} \right),
\end{align}

we have

\begin{align}
  \nabla_k \bullet \nabla_k f_{ee} = \sum_{l=1}^{N_e}\frac{\partial}{\partial \mathbf{r}_k} 
    \bullet \left( \hat{r}_{kl} g(r_{kl}) \right)
    = \sum_{l=1}^{N_e}(g(r_{kl})\nabla_k \bullet \hat{r}_{kl} + \hat{r}_{kl} \bullet \nabla_k g(r_{kl})).
\end{align}

The gradient of $g(r_{kl})$ yields

\begin{align}
  \nabla_k g(r_{kl}) = &-\kappa e^{-\kappa r_{kl}}\left( 
    \frac{b_1}{(1 + b_2 R_{kl})^2} + \sum_{p=2}^{N_b - 1} p b_{p+1} R_{kl}^{p-1}
    \right)\hat{r}_{kl}\\
    & + \kappa e^{-2\kappa r_{kl}}\left( 
    \frac{-2b_1b_2}{(1 + b2R_{kl})^3} + \sum_{p=2}^{N_b - 1}p(p-1)b_{p+1}R_{kl}^{p-2} 
    \right)\hat{r}_{kl}.
\end{align}

The divergence of $\hat{r}_{kl}$ is simply $2/r_{kl}$. This then finally yields

\begin{align}
  \nabla_k \bullet (\hat{r}_{kl}g(r_{kl})) &= g(r_{kl})\left(\frac{2}{r_{kl}} - \kappa\right)\\
    &+ \kappa e^{-2\kappa r_{kl}}\left( 
      \frac{-2b_1b_2}{(1 + b_2R_{kl})^3} + \sum_{p=2}^{N_b - 1} p(p-1)b_{p+1}R_{kl}^{p-2}  
    \right),
\end{align}

from which we can find the laplacian of $f_{ee}$ as

\begin{align}
  \nabla^2 f_{ee} = \sum_{k=1}^{N_e} \sum_{l \neq k}^{N_e} \nabla_k 
    \cdot (\hat{r}_{kl}g(r_{kl})).
\end{align}

\end{document}
